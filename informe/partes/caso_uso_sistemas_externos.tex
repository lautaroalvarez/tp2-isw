\subsection{Interacciones con instrumentos externos o iniciadas por el sistema}

\par En esta subsección vamos a mencionar y dar una explicación general de las interacciones entre el sistema ARS y distintos instrumentos de medición y sistemas externos. También nombraremos el envío de avisos de nuevos datos a entres reguladores.

\par Antes que nada, vamos a enumerar los actores que participan y explicar un poco de qué se tratan:
\begin{itemize}
  \item \textbf{Sensor de temperatura} y \textbf{Sensor de presión}: Se trata de equipos físicos externos que informarán al sistema sobre distintos datos de temperatura y presión del simulador SimOil.
  \item \textbf{SimOil}: Se trata de un módulo del simulador SimOil con el cual el sistema interactuará para pedirle datos que crea necesarios para detectar anomalías.
  \item \textbf{Encargado de SMS}: Se trata de un módulo de un sistema externo con el cual el sistema ARS interactuará para pedirle que realice el envío de mensajes SMS.
  \item \textbf{Ente regulador}: Se trata de un módulo de un sistema propio del Ente reguladora con el cual el sistema ARS interactuará para dar aviso de nuevos datos capturados.
\end{itemize}

\par En el gráfico de la figura \ref{fig:casosuso_sistemas} podemos ver cómo los distintos actores se relacionan con los siguientes casos de uso:

\begin{itemize}
  \item \textbf{Recibiendo medición de temperatura}: El actor \textit{Sensor de temperatura} envía al sistema ARS un dato de temperatura capturado en el SimOil. Al recibirlo, el sistema lo almacena para luego ser procesado por distintos módulos y da aviso al sensor de que los datos fueron recibidos corréctamente.
  \item \textbf{Recibiendo medición de presión}: El actor \textit{Sensor de presión} envía al sistema ARS un dato de presión tomado del SimOil y espera el aviso de recepción por parte del sistema. Al igual que en el caso anterior, el sistema simplemente lo almacena para ser procesado mas tarde.
  \item \textbf{Buscando anomalías}: Previo a realizar distintos procesos de detección de anomalías, el sistema debe contar con distintos datos del estado actual del simulador SimOil. Por esto, envía uno o mas pedidos de información al actor \textbf{SimOil}. Este simplemente devuelve la información solicitada. Luego de esto, continúa normalmente la búsqueda de anomalías.
\end{itemize}

\begin{figure}[ht]
  \center
  \begin{tikzpicture}
    \begin{umlsystem}[x=0, y=0, fill=green!10]{ARS}
      %--casos de uso
      \umlusecase[x=0, y=0, width=2cm, name=RecibiendoMedicionDeTemperatura]{Recibiendo medición de temperatura}
      \umlusecase[x=3, y=0, width=1.5cm, name=BuscandoAnomalias]{Buscando Anomalías}
      \umlusecase[x=0, y=-2, width=1.5cm, name=ActivandoAlarma]{Activando Alarma}
      \umlusecase[x=3, y=-2, width=2cm, name=RecibiendoMedicionDePresion]{Recibiendo medición de presión}
      \umlusecase[x=1.5, y=-4, width=2cm, name=InformandoQueHayNuevosDatos]{Informando que hay nuevos datos}
    \end{umlsystem}

    %--actores
    \umlactor[x=-4, y=0]{Sensor de temperatura}
    \umlactor[x=7, y=0]{SimOil}
    \umlactor[x=7, y=-2]{Sensor de presion}
    \umlactor[x=-4, y=-2]{Encargado de SMS}
    \umlactor[x=7, y=-4]{Ente regulador}

    %--relacion actores - casos de uso
    \umlassoc{Sensor de presion}{RecibiendoMedicionDePresion}
    \umlassoc{Sensor de temperatura}{RecibiendoMedicionDeTemperatura}
    \umlassoc{SimOil}{BuscandoAnomalias}
    \umlassoc{Encargado de SMS}{ActivandoAlarma}
    \umlassoc{Ente regulador}{InformandoQueHayNuevosDatos}
  \end{tikzpicture}
  \caption{Gráfico de casos de uso relacionados con sistemas e instrumentos externos y otros iniciados por el sistema.}
  \label{fig:casosuso_sistemas}
\end{figure}
