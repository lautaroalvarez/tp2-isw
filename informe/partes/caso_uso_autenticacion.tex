\subsection{Autenticación}
\label{sec:casosuso_autenticacion}

\par Antes que nada, vamos a definir los casos de uso relacionados con la autenticación del usuario en el sistema. En la figura \ref{fig:casosuso_autenticacion} podemos que simplemente tenemos un actor \textit{Usuario} que es quien participa de ambos casos de uso.

\begin{figure}[ht]
  \center
  \begin{tikzpicture}
    \begin{umlsystem}[y=1, x=2, fill=green!10]{ARS}
      % casos de uso
      \umlusecase[y=0, name=Autenticadonse]{Autenticándose}
      \umlusecase[y=-2, width=2.5cm, name=CancelandoAutenticacion]{Cancelando autenticación}
    \end{umlsystem}

    % actores
    \umlactor[x=-2, y=0]{Usuario}

    % relaciones actor - caso de uso
    \umlassoc{Usuario}{Autenticadonse}
    \umlassoc{Usuario}{CancelandoAutenticacion}
  \end{tikzpicture}
  \caption{Gráfico de casos de uso relacionados con autenticación y actores que intervienen.}
  \label{fig:casosuso_autenticacion}
\end{figure}

\par A continuación pasaremos a explicar los dos casos de uso:
\begin{itemize}
  \item \textbf{Autenticándose:} El usuario ingresa sus datos de cuenta y es autenticado por el sistema, quien le da unas credenciales correspondientes a los roles que tiene permitidos asumir.
  \item \textbf{Cancelando autenticación:} El sistema bloquea las credenciales que había otorgado al usuario y no le permite mas el acceso a los módulos del sistema que requieran permisos.
\end{itemize}
