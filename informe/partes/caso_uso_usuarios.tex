\subsection{Roles de usuarios e interacción con el sistema}

\par En todos los casos de uso que mencionaremos aquí, se requiere que el actor se haya autenticado previamente. Esta autenticación debió ser otorgada por el sistema al usuario como es mencionado en la subsección \ref{sec:casosuso_autenticacion}. Como vemos en el gráfico de casos de uso de la figura \ref{fig:casosuso_usuarios}, los distintos actores heredan de \textit{Usuario}.

\par Podemos diferenciar tres actores: \textbf{Visualizador de monitoreo}, \textbf{Visualizador de eventos} y \textbf{Encargado de alarmas}. Cada uno de ellos, puede realizar interacciones con el sistema. Vamos a mencionarlas todas y dar una explicación general,  mas adelante (en la subsección \ref{sec:casosuso_cuadros}) se seleccionarán las mas importantes y se detallará su funcionamiento en una tabla \textit{curso normal / curso alternativo}.

\begin{itemize}
  \item \textbf{Viendo sistema de monitoreo}: El actor \textit{Visualizador de monitoreo} ingresa al sistema y visualiza un listado de datos recolectados por los sensores y también datos que ya fueron procesados y analizados.
  \item \textbf{Viendo listado de eventos}: El actor \textit{Visualizador de eventos} ingresa al sistema y puede observar el listado de eventos anómalos detectados. Aquí podrá hacer click sobre alguno en particular para pasar a \textit{Viendo un evento}.
  \item \textbf{Viendo un evento}: El actor \textit{Visualizador de eventos} puede observar aquí los datos de un evento en particular (yacimiento, pozo e instrumento o sensor que produjo la activación de la alarma), el estado y la acción correctiva tomada por el Jefe de operaciones en caso de ser un incoveniente real o de lo contrario, simplemente indicar la falsa alarma.
  \item \textbf{Resolviendo evento de alarma}: El actor \textit{Encargado de alarmas} ingresa al sistema luego de haber recibido una notificación de alarma para completar los datos faltantes del formulario de evento. Entre otras cosas, debe indicar al sistema qué acción correctiva se tomó o si se trata de una falsa alarma.
\end{itemize}

\begin{figure}[ht]
  \center
  \begin{tikzpicture}
    \begin{umlsystem}[x=9, y=-1, fill=green!10]{ARS}
      %--casos de uso
      \umlusecase[x=0, y=0, width=3cm, name=ViendoSistemaDeMonitoreo]{Viendo sistema de monitoreo}
      \umlusecase[x=2, y=-2, width=1.5cm, name=ViendoUnEvento]{Viendo un evento}
      \umlusecase[x=0, y=-4, width=3cm, name=ViendoListadoDeEventos]{Viendo listado de eventos}
      \umlusecase[x=0, y=-6, width=3cm, name=ResolviendoEventoDeAlarma]{Resolviendo evento de alarma}
    \end{umlsystem}

    %--relacion entre casos de uso
    \umlVHextend{ViendoListadoDeEventos}{ViendoUnEvento}

    %--actores
    \umlactor[x=3, y=-1]{Visualizador de monitoreo}
    \umlactor[x=4, y=-5]{Visualizador de eventos}
    \umlactor[x=3, y=-7]{Encargado de alarmas}
    \umlactor[x=0, y=-4]{Usuario}

    %--herencia de actores
    \umlinherit{Usuario}{Visualizador de monitoreo}
    \umlinherit{Usuario}{Visualizador de eventos}
    \umlinherit{Usuario}{Encargado de alarmas}

    %--relacion actores - casos de uso
    \umlassoc{Visualizador de monitoreo}{ViendoSistemaDeMonitoreo}
    \umlassoc{Visualizador de eventos}{ViendoListadoDeEventos}
    \umlassoc{Encargado de alarmas}{ResolviendoEventoDeAlarma}
  \end{tikzpicture}
  \caption{Gráfico de casos de uso relacionados con usuarios accediento a la interfaz del sistema ARS.}
  \label{fig:casosuso_usuarios}
\end{figure}
