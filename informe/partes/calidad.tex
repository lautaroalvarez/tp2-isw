\section{Atributos de Calidad}

\subsection{Disponibilidad}

Backup en caso de pérdida o siniestro
\begin{itemize}
\item {\bf Fuente}: Vulnerabilidad en el sistema
\item {\bf Estímulo}: Daño directo en los datos almacenados
\item {\bf Artefacto}: Base de datos
\item {\bf Entorno}: Sistema en estado normal
\item {\bf Respuesta}: Loguear falla y restaurar último backup
\item {\bf Medición}: No crashear el sistema al no poder acceder a la BD y recuperar el estado normal del sistema en menos de 15 segundos 
\end{itemize}

Tolerancia a fallas
\begin{itemize}
\item {\bf Fuente}: Sensor
\item {\bf Estímulo}: Fuera de funcionamiento
\item {\bf Artefacto}: Interfaz de sensores
\item {\bf Entorno}: Sistema en estado normal
\item {\bf Respuesta}: Se deshabilita el sensor para no crashear el sistema
\item {\bf Medición}: El sistema continúa en ejecución frente a problemas externos de los sensores
\end{itemize}

La activación de la alarma implica dar aviso en forma inmediata al Jefe de Operaciones del yacimiento via un servicio de sms
\begin{itemize}
\item {\bf Fuente}: Encargado de SMS
\item {\bf Estímulo}: notificación de SMS no enviado
\item {\bf Artefacto}: Interfaz de envío de SMS
\item {\bf Entorno}: Sistema en estado de alerta
\item {\bf Respuesta}: Se gestiona algún mecanismo para repetir el envío del mensaje hasta lograrlo con éxito
\item {\bf Medición}: El sistema continúa en funcionamiento durante el estado de alarma y a su vez no debe dejar de intentar el envío de SMS hasta no recibir el aviso de envío exitoso
\end{itemize}

\subsection{Performance}

El sistema no debe tener ningún tipo de demoras
\begin{itemize}
\item {\bf Fuente}: Jefe de Operaciones
\item {\bf Estímulo}: Alta de un evento por resolución de incidente
\item {\bf Artefacto}: UI del ABM
\item {\bf Entorno}: Sistema en estado normal, con sobrecarga de usuarios de consulta, provenientes del Ministerio y Entes Reguladores
\item {\bf Respuesta}: Se gestiona algún mecanismo para tomar la operación realizada por el Jefe de Operaciones, evitando que el mismo perciba algún tipo de demora
\item {\bf Medición}: El sistema no tarda mas de dos segundos en resolver la operación para el 95 por ciento de las consultas
\end{itemize}

La activación de la alarma implica dar aviso en forma inmediata al Jefe de Operaciones del yacimiento via un servicio de sms
\begin{itemize}
\item {\bf Fuente}: Módulo de detección de anomalías
\item {\bf Estímulo}: Detección de anomalía
\item {\bf Artefacto}: Interfaz de envío de SMS
\item {\bf Entorno}: Sistema en estado normal, con sobrecarga de usuarios de consulta, provenientes del Ministerio y Entes Reguladores. Los datos de contacto son correctos, el celular destinatario tiene señal y el servicio de SMS está activo
\item {\bf Respuesta}: Se gestiona algún mecanismo para ejecutar el envío del mensaje lo antes posible, con los recursos necesarios para evitar cualquier tipo de demora
\item {\bf Medición}: El sistema envía el mensaje en menos de 2 segundos después de la detección de la anomalía y recibe la notificación de recepción de SMS en menos de 10 segundos mas 
\end{itemize}

Manejo de grandes volúmenes de datos
\begin{itemize}
\item {\bf Fuente}: Usuario Visualizador de Eventos
\item {\bf Estímulo}: Accede a ver un evento
\item {\bf Artefacto}: UI e interfaz con BD
\item {\bf Entorno}: Sistema en estado normal, con sobrecarga de usuarios de consulta, provenientes del Ministerio y Entes Reguladores.
\item {\bf Respuesta}: Se muestra el evento requerido con todos sus datos correctos
\item {\bf Medición}: El sistema envía muestra la información solicitada en menos de 5 segundos en el 95 por ciento de los casos
\end{itemize}

\subsection{Usabilidad}

Manejo de grandes volúmenes de datos
\begin{itemize}
\item {\bf Fuente}: Usuario Visualizador de Eventos
\item {\bf Estímulo}: Accede a ver un evento
\item {\bf Artefacto}: UI e interfaz con BD
\item {\bf Entorno}: Sistema en estado normal, con sobrecarga de usuarios de consulta, provenientes del Ministerio y Entes Reguladores.
\item {\bf Respuesta}: Se muestra una imagen de carga del estilo "reloj de arena", mientras se cargan los datos solicitados
\item {\bf Medición}: El porcentaje de usuarios que abandonan la ventana es menor al 5 por ciento
\end{itemize}

Acceso al sistema de monitoreo/informes desde el Ministerio/Entes Reguladores
\begin{itemize}
\item {\bf Fuente}: Usuario Visualizador
\item {\bf Estímulo}: Accede al sistema desde su lugar de trabajo
\item {\bf Artefacto}: Módulo de Autenticación
\item {\bf Entorno}: Sistema en estado normal, se ingresan datos válidos de logueo
\item {\bf Respuesta}: Se loguea el usuario indicado y se muestra el panel o dashboard que le corresponde al rol vinculado con su usuario, con el fin de optimizar la navegación por el sistema 
\item {\bf Medición}: Tiempo de navegación dentro del sistema
\end{itemize}

\subsection{Testeabilidad}

[El proceso de Limpieza] puede remover aproximadamente un 80 por ciento de estas anomalías de la mayoría de los conjuntos de datos
\begin{itemize}
\item {\bf Fuente}: Set de valores con anomalías introducidas de manera controlada
\item {\bf Estímulo}: Ejecución de sistema con dichos valores
\item {\bf Artefacto}: Módulo de detección de anomalías
\item {\bf Entorno}: Sistema en estado normal, en desarrollo
\item {\bf Respuesta}: Se registran los valores aceptados "en crudo" y aquellos que fueron filtrados en la limpieza, en la BD, por cada set de datos
\item {\bf Medición}: Se compara en cada set, los datos filtrados por sobre el total de anomalías insertadas. El porcentaje debe ser aproximadamente del 80 por ciento, con 5 por ciento de margen de error.
\end{itemize}