\newpage
\section{Casos de Uso}

\subsection{Autenticación}

\par Antes que nada, vamos a definir los casos de uso relacionados con la autenticación del usuario en el sistema.

\begin{tikzpicture}
  \begin{umlsystem}[y=1, x=2, fill=green!10]{ARS}
    % casos de uso
    \umlusecase[y=0, name=Autenticadonse]{Autenticándose}
    \umlusecase[y=-2, width=2.5cm, name=CancelandoAutenticacion]{Cancelando autenticación}
  \end{umlsystem}

  % actores
  \umlactor[x=-2, y=0]{Usuario}

  % relaciones actor - caso de uso
  \umlassoc{Usuario}{Autenticadonse}
  \umlassoc{Usuario}{CancelandoAutenticacion}
\end{tikzpicture}

\subsection{Roles de usuarios e interacción con el sistema}

\par A continuación vamos a mencionar y analizar los casos de uso que se llevan a cabo por interacción del usuario con una interfaz web. En todos los casos de uso que mencionaremos aquí, se requiere al actor previa autenticación. Esta autenticación debió ser otorgada por el sistema al usuario como es mencionado en la subsección \ref{sec:casosuso_autenticacion}.

\par Como vemos en el gráfico de casos de uso de la figura \ref{fig:casosuso_usuarios}, los distintos actores heredan de \textit{Usuario} los casos de uso relacionados con autenticación.

\begin{figure}[ht]
  \center
  \begin{tikzpicture}
    \begin{umlsystem}[x=9, y=-1, fill=green!10]{ARS}
      %--casos de uso
      \umlusecase[x=0, y=0, width=3cm, name=ViendoSistemaDeMonitoreo]{Viendo sistema de monitoreo}
      \umlusecase[x=2, y=-2, width=1.5cm, name=ViendoUnEvento]{Viendo un evento}
      \umlusecase[x=0, y=-4, width=3cm, name=ViendoListadoDeEventos]{Viendo listado de eventos}
      \umlusecase[x=0, y=-6, width=3cm, name=ResolviendoEventoDeAlarma]{Resolviendo evento de alarma}
    \end{umlsystem}

    %--relacion entre casos de uso
    \umlVHextend{ViendoListadoDeEventos}{ViendoUnEvento}

    %--actores
    \umlactor[x=3, y=-1]{Visualizador de monitoreo}
    \umlactor[x=4, y=-5]{Visualizador de eventos}
    \umlactor[x=3, y=-7]{Encargado de alarmas}
    \umlactor[x=0, y=-4]{Usuario}

    %--herencia de actores
    \umlinherit{Usuario}{Visualizador de monitoreo}
    \umlinherit{Usuario}{Visualizador de eventos}
    \umlinherit{Usuario}{Encargado de alarmas}

    %--relacion actores - casos de uso
    \umlassoc{Visualizador de monitoreo}{ViendoSistemaDeMonitoreo}
    \umlassoc{Visualizador de eventos}{ViendoListadoDeEventos}
    \umlassoc{Encargado de alarmas}{ResolviendoEventoDeAlarma}
  \end{tikzpicture}
  \caption{Gráfico de casos de uso relacionados con usuarios accediento a la interfaz web del sistema ARS.}
  \label{fig:casosuso_usuarios}
\end{figure}

\subsection{Interacción con sistemas e instrumentos externos}

\par Estos no hace falta que estén logueados. No son usuarios sino `cosas' (explicar bien)

\begin{tikzpicture}
  \begin{umlsystem}[x=0, y=0, fill=green!10]{ARS}
    %--casos de uso
    \umlusecase[x=0, y=0, width=2cm, name=RecibiendoMedicionDeTemperatura]{Recibiendo medición de temperatura}
    \umlusecase[x=3, y=0, width=1.5cm, name=BuscandoAnomalias]{Buscando Anomalías}
    \umlusecase[x=0, y=-2, width=1.5cm, name=ActivandoAlarma]{Activando Alarma}
    \umlusecase[x=3, y=-2, width=2cm, name=RecibiendoMedicionDePresion]{Recibiendo medición de presión}
    \umlusecase[x=1.5, y=-4, width=2cm, name=InformandoQueHayNuevosDatos]{Informando que hay nuevos datos}
  \end{umlsystem}

  %--actores
  \umlactor[x=-4, y=0]{Sensor de temperatura}
  \umlactor[x=7, y=0]{SimOil}
  \umlactor[x=7, y=-2]{Sensor de presion}
  \umlactor[x=-4, y=-2]{Encargado de SMS}
  \umlactor[x=7, y=-4]{Ente regulador}

  %--relacion actores - casos de uso
  \umlassoc{Sensor de presion}{RecibiendoMedicionDePresion}
  \umlassoc{Sensor de temperatura}{RecibiendoMedicionDeTemperatura}
  \umlassoc{SimOil}{BuscandoAnomalias}
  \umlassoc{Encargado de SMS}{ActivandoAlarma}
  \umlassoc{Ente regulador}{InformandoQueHayNuevosDatos}
\end{tikzpicture}


\subsection{Exlicación en detalle de los casos mas importantes}
\label{sec:casosuso_cuadros}

\par A continuación, tomaremos 3 casos de uso y pasaremos a explicarlos en detalle. Los casos de uso seleccionados fueron los que nos parecieron mas importantes y que mas aportaban al entendimiento del sistema general.

\begin{casodeuso}
  \cutitle{Resolviendo un evento de alarma}
  \cuactors{}
  \cupre{El sistema debe encontrarse en modo alarma como consecuencia del evento que el usuario desea resolver}
  \cupost{El sistema pasa a estar nuevamente en estado normal y el evento se marcará como resuelto.}
  \cucourse{
    1) El sistema carga la información del evento: Yacimiento o pozo involucrado, el o los instrumentos de medición responsables, fecha y hora de la medición, entre otros datos. & \\
    2) El sistema muestra al usuario dos botones para seleccionar si se trata de una falsa alarma o si no. &
      2.1) Si el usuario indica que se trató de una falsa alarma pasa diréctamente a la confirmación de los datos a guardar 5).\\

    3) El usuario completa los campos de la explicación de la acción correctiva tomada: tipo de acción, fecha y hora de la acción, persona encargada de realizar la acción y observaciones adicionales. & \\
    4) El usuario indica que quiere guardar la revisión. & \\
    5) El sistema le consulta al usuario si está seguro que desea guardar la revisión. & 
      5.1) Si el usuario ingresa que no, vuelve a la parte de edición de datos de revisión. \\
    6) El sistema guarda los datos y le muestra al usuario un mensaje de aviso. & \\
    7) Fin del caso. & \\
  }
  \culabel{autenticandose}
\end{casodeuso}