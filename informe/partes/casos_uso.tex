\newpage
\section{Casos de Uso}

\subsection{Autenticación}
\label{sec:casosuso_autenticacion}

\par Antes que nada, vamos a definir los casos de uso relacionados con la autenticación del usuario en el sistema. En la figura \ref{fig:casosuso_autenticacion} podemos que simplemente tenemos un actor \textit{Usuario} que es quien participa de ambos casos de uso.

\begin{figure}[ht]
  \center
  \begin{tikzpicture}
    \begin{umlsystem}[y=1, x=2, fill=green!10]{ARS}
      % casos de uso
      \umlusecase[y=0, name=Autenticadonse]{Autenticándose}
      \umlusecase[y=-2, width=2.5cm, name=CancelandoAutenticacion]{Cancelando autenticación}
    \end{umlsystem}

    % actores
    \umlactor[x=-2, y=0]{Usuario}

    % relaciones actor - caso de uso
    \umlassoc{Usuario}{Autenticadonse}
    \umlassoc{Usuario}{CancelandoAutenticacion}
  \end{tikzpicture}
  \caption{Gráfico de casos de uso relacionados con autenticación y actores que intervienen.}
  \label{fig:casosuso_autenticacion}
\end{figure}

\par A continuación pasaremos a explicar los dos casos de uso:
\begin{itemize}
  \item \textbf{Autenticándose:} El usuario ingresa sus datos de cuenta y es autenticado por el sistema, quien le da unas credenciales correspondientes a los roles que tiene permitidos asumir.
  \item \textbf{Cancelando autenticación:} El sistema bloquea las credenciales que había otorgado al usuario y no le permite mas el acceso a los módulos del sistema que requieran permisos.
\end{itemize}

\subsection{Roles de usuarios e interacción con el sistema}

\par En todos los casos de uso que mencionaremos aquí, se requiere que el actor se haya autenticado previamente. Esta autenticación debió ser otorgada por el sistema al usuario como es mencionado en la subsección \ref{sec:casosuso_autenticacion}. Como vemos en el gráfico de casos de uso de la figura \ref{fig:casosuso_usuarios}, los distintos actores heredan de \textit{Usuario}.

\par Podemos diferenciar tres actores: \textbf{Visualizador de monitoreo}, \textbf{Visualizador de eventos} y \textbf{Encargado de alarmas}. Cada uno de ellos, puede realizar interacciones con el sistema. Vamos a mencionarlas todas y dar una explicación general,  mas adelante (en la subsección \ref{sec:casosuso_cuadros}) se seleccionarán las mas importantes y se detallará su funcionamiento en una tabla \textit{curso normal / curso alternativo}.

\begin{itemize}
  \item \textbf{Viendo sistema de monitoreo}: El actor \textit{Visualizador de monitoreo} ingresa al sistema y visualiza un listado de datos recolectados por los sensores y también datos que ya fueron procesados y analizados.
  \item \textbf{Viendo listado de eventos}: El actor \textit{Visualizador de eventos} ingresa al sistema y puede observar el listado de eventos anómalos detectados. Aquí podrá hacer click sobre alguno en particular para pasar a \textit{Viendo un evento}.
  \item \textbf{Viendo un evento}: El actor \textit{Visualizador de eventos} puede observar aquí los datos de un evento en particular (yacimiento, pozo e instrumento o sensor que produjo la activación de la alarma), el estado y la acción correctiva tomada por el Jefe de operaciones en caso de ser un incoveniente real o de lo contrario, simplemente indicar la falsa alarma.
  \item \textbf{Resolviendo evento de alarma}: El actor \textit{Encargado de alarmas} ingresa al sistema luego de haber recibido una notificación de alarma para completar los datos faltantes del formulario de evento. Entre otras cosas, debe indicar al sistema qué acción correctiva se tomó o si se trata de una falsa alarma.
\end{itemize}

\begin{figure}[ht]
  \center
  \begin{tikzpicture}
    \begin{umlsystem}[x=9, y=-1, fill=green!10]{ARS}
      %--casos de uso
      \umlusecase[x=0, y=0, width=3cm, name=ViendoSistemaDeMonitoreo]{Viendo sistema de monitoreo}
      \umlusecase[x=2, y=-2, width=1.5cm, name=ViendoUnEvento]{Viendo un evento}
      \umlusecase[x=0, y=-4, width=3cm, name=ViendoListadoDeEventos]{Viendo listado de eventos}
      \umlusecase[x=0, y=-6, width=3cm, name=ResolviendoEventoDeAlarma]{Resolviendo evento de alarma}
    \end{umlsystem}

    %--relacion entre casos de uso
    \umlVHextend{ViendoListadoDeEventos}{ViendoUnEvento}

    %--actores
    \umlactor[x=3, y=-1]{Visualizador de monitoreo}
    \umlactor[x=4, y=-5]{Visualizador de eventos}
    \umlactor[x=3, y=-7]{Encargado de alarmas}
    \umlactor[x=0, y=-4]{Usuario}

    %--herencia de actores
    \umlinherit{Usuario}{Visualizador de monitoreo}
    \umlinherit{Usuario}{Visualizador de eventos}
    \umlinherit{Usuario}{Encargado de alarmas}

    %--relacion actores - casos de uso
    \umlassoc{Visualizador de monitoreo}{ViendoSistemaDeMonitoreo}
    \umlassoc{Visualizador de eventos}{ViendoListadoDeEventos}
    \umlassoc{Encargado de alarmas}{ResolviendoEventoDeAlarma}
  \end{tikzpicture}
  \caption{Gráfico de casos de uso relacionados con usuarios accediento a la interfaz del sistema ARS.}
  \label{fig:casosuso_usuarios}
\end{figure}

\subsection{Interacciones con instrumentos externos o iniciadas por el sistema}

\par En esta subsección vamos a mencionar y dar una explicación general de las interacciones entre el sistema ARS y distintos instrumentos de medición y sistemas externos. También nombraremos el envío de avisos de nuevos datos a entres reguladores.

\par Antes que nada, vamos a enumerar los actores que participan y explicar un poco de qué se tratan:
\begin{itemize}
  \item \textbf{Sensor de temperatura} y \textbf{Sensor de presión}: Se trata de equipos físicos externos que informarán al sistema sobre distintos datos de temperatura y presión del simulador SimOil.
  \item \textbf{SimOil}: Se trata de un módulo del simulador SimOil con el cual el sistema interactuará para pedirle datos que crea necesarios para detectar anomalías.
  \item \textbf{Encargado de SMS}: Se trata de un módulo de un sistema externo con el cual el sistema ARS interactuará para pedirle que realice el envío de mensajes SMS.
  \item \textbf{Ente regulador}: Se trata de un módulo de un sistema propio del Ente reguladora con el cual el sistema ARS interactuará para dar aviso de nuevos datos capturados.
\end{itemize}

\par En el gráfico de la figura \ref{fig:casosuso_sistemas} podemos ver cómo los distintos actores se relacionan con los siguientes casos de uso:

\begin{itemize}
  \item \textbf{Recibiendo medición de temperatura}: El actor \textit{Sensor de temperatura} envía al sistema ARS un dato de temperatura capturado en el SimOil. Al recibirlo, el sistema lo almacena para luego ser procesado por distintos módulos y da aviso al sensor de que los datos fueron recibidos corréctamente.
  \item \textbf{Recibiendo medición de presión}: El actor \textit{Sensor de presión} envía al sistema ARS un dato de presión tomado del SimOil y espera el aviso de recepción por parte del sistema. Al igual que en el caso anterior, el sistema simplemente lo almacena para ser procesado mas tarde.
  \item \textbf{Buscando anomalías}: Previo a realizar distintos procesos de detección de anomalías, el sistema debe contar con distintos datos del estado actual del simulador SimOil. Por esto, envía uno o mas pedidos de información al actor \textbf{SimOil}. Este simplemente devuelve la información solicitada. Luego de esto, continúa normalmente la búsqueda de anomalías.
\end{itemize}

\begin{figure}[ht]
  \center
  \begin{tikzpicture}
    \begin{umlsystem}[x=0, y=0, fill=green!10]{ARS}
      %--casos de uso
      \umlusecase[x=0, y=0, width=2cm, name=RecibiendoMedicionDeTemperatura]{Recibiendo medición de temperatura}
      \umlusecase[x=3, y=0, width=1.5cm, name=BuscandoAnomalias]{Buscando Anomalías}
      \umlusecase[x=0, y=-2, width=1.5cm, name=ActivandoAlarma]{Activando Alarma}
      \umlusecase[x=3, y=-2, width=2cm, name=RecibiendoMedicionDePresion]{Recibiendo medición de presión}
      \umlusecase[x=1.5, y=-4, width=2cm, name=InformandoQueHayNuevosDatos]{Informando que hay nuevos datos}
    \end{umlsystem}

    %--actores
    \umlactor[x=-4, y=0]{Sensor de temperatura}
    \umlactor[x=7, y=0]{SimOil}
    \umlactor[x=7, y=-2]{Sensor de presion}
    \umlactor[x=-4, y=-2]{Encargado de SMS}
    \umlactor[x=7, y=-4]{Ente regulador}

    %--relacion actores - casos de uso
    \umlassoc{Sensor de presion}{RecibiendoMedicionDePresion}
    \umlassoc{Sensor de temperatura}{RecibiendoMedicionDeTemperatura}
    \umlassoc{SimOil}{BuscandoAnomalias}
    \umlassoc{Encargado de SMS}{ActivandoAlarma}
    \umlassoc{Ente regulador}{InformandoQueHayNuevosDatos}
  \end{tikzpicture}
  \caption{Gráfico de casos de uso relacionados con sistemas e instrumentos externos y otros iniciados por el sistema.}
  \label{fig:casosuso_sistemas}
\end{figure}


\subsection{Exlicación en detalle de los casos mas importantes}
\label{sec:casosuso_cuadros}

\par A continuación, tomaremos 3 casos de uso y pasaremos a explicarlos en detalle. Los casos de uso seleccionados fueron los que nos parecieron mas importantes y que mas aportaban al entendimiento del sistema general.

\begin{casodeuso}
  \cutitle{Resolviendo un evento de alarma}
  \cuactors{Encargado de alarmas}
  \cupre{El sistema debe encontrarse en modo alarma como consecuencia del evento que el usuario desea resolver.}
  \cupost{El sistema pasa a estar nuevamente en estado normal y el evento se marcará como resuelto.}
  \cucourse{
    1) El sistema carga la información del evento: Yacimiento o pozo involucrado, el o los instrumentos de medición responsables, fecha y hora de la medición, entre otros datos. & \\
    2) El sistema muestra al usuario dos botones para seleccionar si se trata de una falsa alarma o si no. &
      2.1) Si el usuario indica que se trató de una falsa alarma pasa diréctamente a la confirmación de los datos a guardar 5).\\

    3) El usuario completa los campos de la explicación de la acción correctiva tomada: tipo de acción, fecha y hora de la acción, persona encargada de realizar la acción y observaciones adicionales. & \\
    4) El usuario indica que quiere guardar la revisión. & \\
    5) El sistema le consulta al usuario si está seguro que desea guardar la revisión. &
      5.1) Si el usuario ingresa que no, vuelve a la parte de edición de datos de revisión. \\
    6) El sistema guarda los datos y le muestra al usuario un mensaje de aviso. & \\
    7) Fin del caso. & \\
  }
  \culabel{resoviendo_alarma}
\end{casodeuso}

\begin{casodeuso}
  \cutitle{Activando Alarma}
  \cuactors{Encargado de SMS}
  \cupre{El sistema debe encontrarse en modo normal y se tuvo que haber encontrado una anomalía.}
  \cupost{El sistema pasa a estar en modo alarma y se realiza el envío de SMS con éxito.}
  \cucourse{
    1) El sistema pasa a estar en modo alarma. & \\
    2) El sistema crea un evento con los datos de la anomalía detectada. & \\
    3) El sistema busca los datos del servidor de envío de SMS y los datos del encargado de alarmas. & \\
    4) El sistema manda al \textit{Encargado de SMS} la petición para enviar un mensaje al encargado de alarmas. & \\
    5) El sistema espera la confirmación del \textit{Encargado de SMS}. &
      5.1) Si se supera el tiempo de espera máximo, se espera un tiempo determinado y se reenvía la petición al servidor 4) \\
    6) Fin del caso. & \\
  }
  \culabel{activando_alarma}
\end{casodeuso}

\begin{casodeuso}
  \cutitle{Recibiendo medición de temperatura}
  \cuactors{Sensor de Temperatura}
  \cupre{}
  \cupost{El dato de medición se almacena con éxito y se confirma al sensor la recepción del dato.}
  \cucourse{
    1) El \textit{Sensor de Temperatura} envía un dato de medición al sistema. & \\
    2) El sistema toma el dato recibido y lo almacena para luego ser procesado. & \\
    3) El sistema envía un aviso de recepción exitosa al \textit{Sensor de Temperatura}, para que este no reenvíe los datos. & \\
    6) Fin del caso. & \\
  }
  \culabel{recibiendo_temperatura}
\end{casodeuso}

\begin{casodeuso}
  \cutitle{Autenticación de un usuario}
  \cuactors{Usuario sin identificar}
  \cupre{El usuario no debe estar autenticado y debe estar registrado en el sistema}
  \cupost{El usuario ingresa al sistema}
  \cucourse{
    1) El usuario ingresa a la pantalla de inicial del sistema en la cual figura un formulario con dos campos, uno para el nombre de usuario y otro para la contraseña. & \\
    2) El usuario completa el formulario y apreta un botón que dice ''Ingresar''. & 
        2.1) Si los datos son incorrectos se le impide el ingreso al usuario, permanece en la misma pantalla pero con un cartel que indica que el ingreso no fue autorizado por escribir credenciales incorrectas. Se vuelve al paso 1.& \\
    3) El usuario ingresa a la pantalla correspondiente al tipo de usuario. & \\
    4) Fin de caso de uso. & \\
  }
\end{casodeuso}